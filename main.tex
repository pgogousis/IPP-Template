%
%                       This is a basic LaTeX Template
%                       for the Informatics Research Review

\documentclass[a4paper,11pt]{article}
% Add local fullpage and head macros
\usepackage{head,fullpage}     
% Add graphicx package with pdf flag (must use pdflatex)
\usepackage[pdftex]{graphicx}  
% Better support for URLs
\usepackage{url}
% Date formating
\usepackage{datetime}
% For Gantt chart
\usepackage{pgfgantt}
\usepackage{xcolor}
\usepackage[utf8]{inputenc}
\usepackage{hyperref}
\usepackage{float}
\usepackage{titlesec}

\titlespacing\section{0pt}{12pt plus 4pt minus 2pt}{0pt plus 2pt minus 2pt}
\titlespacing\subsection{0pt}{12pt plus 4pt minus 2pt}{0pt plus 2pt minus 2pt}
\titlespacing\subsubsection{0pt}{12pt plus 4pt minus 2pt}{0pt plus 2pt minus 2pt}

\definecolor{barblue}{RGB}{153,204,254}
\definecolor{groupblue}{RGB}{51,102,254}
\definecolor{linkred}{RGB}{165,0,33}

\newdateformat{monthyeardate}{%
  \monthname[\THEMONTH] \THEYEAR}

\parindent=0pt          %  Switch off indent of paragraphs 
\parskip=5pt            %  Put 5pt between each paragraph  
\Urlmuskip=0mu plus 1mu %  Better line breaks for URLs


%                       This section generates a title page
%                       Edit only the following three lines
%                       providing your exam number, 
%                       the general field of study you are considering
%                       for your review, and name of IRR tutor

\newcommand{\examnumber}{B138641}
\newcommand{\field}{Benchmarking Graph Processing Systems}
\newcommand{\tutor}{Pablo León-Villagrá}
\newcommand{\supervisor}{Dr Milos Nikolic}

\begin{document}
\begin{minipage}[b]{110mm}
        {\Huge\bf School of Informatics
        \vspace*{17mm}}
\end{minipage}
\hfill
\begin{minipage}[t]{40mm}               
        \makebox[40mm]{
        \includegraphics[width=40mm]{crest.png}}
\end{minipage}
\par\noindent
    % Centre Title, and name
\vspace*{2cm}
\begin{center}
        \Large\bf Informatics Project Proposal \\
        \Large\bf \field
\end{center}
\vspace*{1.5cm}
\begin{center}
        \bf \examnumber\\
        \monthyeardate\today
\end{center}
\vspace*{5mm}

%
%                       Insert your abstract HERE
%                       
\begin{abstract}

The aim of this project is to examine state-of-the-art graph processing systems and compare their performance and scalability in both local and parallel settings. More specifically, the project focuses on showing that oftentimes scalability comes at a cost, and in many cases such an approach is not worth the time and effort. In our research we will prove this by conducting several experiments that will demonstrate that a single-node set-up outperforms executions in multi-node environments due to the parallelization overhead that incurs.

\end{abstract}

\vspace*{1cm}

\vspace*{3cm}
Date: \today

\vfill
{\bf Tutor:} \tutor\\
{\bf Supervisor:} \supervisor
\thispagestyle{empty}

\newpage

%                                               Through page and setup 
%                                               fancy headings
\setcounter{page}{1}                            % Set page number to 1
\footruleheight{1pt}
\headruleheight{1pt}
\lfoot{\small School of Informatics}
\lhead{Informatics Research Review}
\rhead{- \thepage}
\cfoot{}
\rfoot{Date: \date{\today}}
%
\tableofcontents % Makes Table of Contents

% \newpage

\section{Introduction} \label{introduction}

% Introduce the topic of research and explain its academic and industrial context.

% \par Modern computer applications and services have evolved to an extent that they have become difficult to manage. The amount of data handled in such applications is only growing while the industry and the research community have built new systems and infrastructures to be able to keep up with this growth rate. Additionally, applications and services now feature relational data, which are often expressed as graph structures, fact that makes things even more demanding and complicated.

\par Graphs have been extensively used in numerous applications and problems, such as the web, social networks, navigation, recommendation systems, and others, while the interest in the domain of graph processing and graph data analysis is constantly growing. This is based on the fact that more and more systems have been developed to process graphs and perform various graph analysis tasks in an efficient manner, such as graph processing systems \cite{naiad, timelydf,graphlab, flink, spark, graphxpaper, powergraph,pregel}, and graph database systems \cite{emptyheaded, neo4j, arangodb, orientdb}.

\par Although these systems have been extensively used in various applications, there have been concerns regarding some of their aspects. Firstly, many people argue that their performance is questionable and that there is a massive overhead that incurs due to the way these systems scale \cite{case-against, scalability-cost, challenges}. Afterwards, there is no clear indication of which system is appropriate for each occasion. For this reason, there have been many researchers focusing on creating benchmarking systems in order to evaluate the performance of such systems and observe their strengths and weaknesses \cite{benchmarking-vision, how-well, graphalytics, traversal-benchmarking}.

\par While research on benchmarking graph processing systems has provided information about the performance of these graph processing systems, it tends to be quite specific and concerns certain aspects of these frameworks. This project was motivated by the need for a more thorough investigation that covers a broader range of cases and systems. Our research aims at showing that scalability is not always the best approach by performing various experiments under different conditions, that is, measuring performance using many different algorithms, systems, and other parameters. The main goal is to demonstrate how in many cases running such tasks with graph processing systems in single-node settings (execution on one physical machine, e.g. a laptop) is more efficient performance-wise than distributed and multi-node settings (execution on multiple physical machines, e.g. a distributed system, or a cluster).

\par In what follows we modularise our problem and briefly mention some of its aspects (Sections 1.1 through 1.6). Next, in \hyperref[background]{Section 2}, we discuss necessary and relevant knowledge to the problem's domain. Afterwards, we consider specific parts of our research and analyse steps in \hyperref[methodology]{Section 3}. Subsequently, we focus on the evaluation of our results which will be detailed in \hyperref[evaluation]{Section 4}, and we highlight what we expect to see in our results in \hyperref[outcomes]{Section 5}. Finally, in \hyperref[milestones]{Section 6}, we state a detailed report of the project's plan, its objectives and the deliverables.

% \begin{itemize}
%     \item Establish the general subject area.
%     \item Describe the broad foundations of your study -- provide adequate background for readers.
%     \item Indicate the general scope of your project.
%     \item Provide an overview of the sections that will appear in your proposal (optional).
%     \item Engage the readers.
% \end{itemize}

\subsection{Problem Statement} \label{problem-statement}

\par While there is research on individual graph processing systems that measures how these systems perform, there has not been extensive work exposing a more objective benchmark for these systems' performance; such a benchmark would feature the use of a wide variety of algorithms, multiple set-ups ranging from one to many cores, as well as more realistic use cases. Our work focuses on taking into account all these parameters and aims to provide a broader benchmark that will be capable of better indicating if such systems are worth using and, if so, under which settings.

% \begin{itemize}
%     \item Answer the question:''What is the gap that needs to be filled?"
%     and/or ''What is the problem that needs to be solved?"
%     \item State the problem clearly early in a paragraph.
%     \item Limit the variables you address in stating your problem.
%     \item Consider bordering the problem as a question.
% \end{itemize}

\subsection{Research Hypothesis and Objectives} \label{hypothesis}

\par Our hypothesis is that graph processing systems do not scale as well as they claim to do for common tasks. We were able to identify these common tasks and other useful information from a survey on graph processing \cite{survey}. Moreover, this survey revealed how the graph research community and the industry perform their analysis tasks, as well as what data they use, what kind of algorithms they run, and so forth.

\par We aim to support our hypothesis by conducting experiments that will demonstrate how the selected graph processing systems perform in single-node settings, as well as when scaled up, and hope to show through our results that for common tasks executions in single-node settings perform better.

% Identify the overall aims of the project and the individual measurable objectives against which you would wish the outcome of the work to be assessed. Clearly spell out any research hypothesis you are following.

% Include a justification (rationale) for the study. Be clear about what your study will not address.

\subsection{Timeliness and Novelty}

\par The graph processing domain is constantly growing and becoming more popular, given that more and more researchers and companies analyse graph data and perform graph processing tasks \cite{facebook-trillion, twitter-sentiment}. Consequently, carrying out this research at this point is crucial since it will provide insights and indications as to how analyses and graph processing tasks can be performed, which will greatly impact the way researchers and developers approach their problems.

% Explain why the proposed research is of sufficient timeliness and novelty

\subsection{Significance}

% Having an indication as to what resources an individual may need before performing their graph data analysis tasks or research is of great significance. 


\par Possessing insights and information related to various aspects of a researcher's or developer's approach for graph data processing tasks is of great importance. This is so, because insights and information help facilitate better (in terms of performance) and less costly (in terms of resources used) approaches. We aim to conduct a thorough experimentation that will cover widely used systems and processing tasks, as reported in \cite{survey}, some of which are likely to be similar to the intended research or analysis task. As such, our work will extend current research by yielding deeper insights, and will help in the process of the decision making with regards to the technologies that will be used, the required resources, the expected performance, and other aspects, which in turn will result in better approaching a problem and providing more timely and efficient solutions.


% The proposal should demonstrate the originality of your intended research. You should therefore explain why your research is important (for example, by explaining how your research builds on and adds to the current state of knowledge in the field or by setting out reasons why it is timely to research your proposed topic) and providing details of any immediate applications, including further research that might be done to build on your findings.

\subsection{Feasibility} \label{feasibility}

\par Firstly, an important factor to take into account regards the configurations of the graph processing systems used. In detail, while the single-node approach is fairly easy to implement, the scaled versions require more attention, as they demand more resources. The difficulty we might face concerns our resources, given that as it is, they are limited to 8 cores. Obtaining access to a cluster environment or some computational nodes will significantly help our research, although, in case such a thing cannot be arranged, our work will be limited to single-node experimentation.

\par Afterwards, a secondary goal of our work is to indicate how some tasks can be carried out with the use of higher-level systems, such as graph database systems \cite{emptyheaded, emptyheadedgit, neo4j}.  Examining such systems and comparing results against graph processing systems will also provide very useful information, and will result in our work covering a broader range of cases, which of course also increases the size of our audience. This part of our research will be carried out in case we manage to complete our main goal within the expected time frame. Therefore, the limiting factor to this secondary task is whether or not we have enough time left once we complete our primary objective.



% \par As mentioned in \hyperref[introduction]{Section 1}, a secondary objective of our work will be to provide alternatives to approaches using less primitive systems, such as graph database systems. More specifically, graph processing has become more and more popular and has gained a lot of attention over the past few years. The people that get involved with such tasks are not necessarily researchers of this field or experienced programmers, so the use of these primitive systems can be quite challenging. For this reason, many people use alternatives, instead, such as graph database systems or querying systems to conduct their work. Thus, examining such systems and comparing results against graph processing systems will also provide very useful information and will result to our work covering a broader range of cases, which of course also grows the size of our audience.

% Comment on the feasibility of the research plans given its limited time frame and resources. Outline your plans for a feasibility study before starting e.g.\ major implementation work.

\subsection{Beneficiaries}

\par Our work will impact the graph research community, as well as the industry, where developers and analysts use graph processing systems to perform their data analysis tasks. In detail, given that the results will provide insight as to whether (i) a scaled approach involving distributed graph processing systems is needed, or if (ii) a more simple, single-node, implementation is more preferable, researchers and small to medium sized companies can benefit from possessing such information because it can potentially result in them saving time, effort, and money.

\par This is due to the fact that scaled approaches demand more resources, which often come at a big cost. Additionally, setting up and managing these resources requires a lot of time and effort. Therefore, having a-priori knowledge about the resources needed for a specific task, or knowing the performance of an algorithm that might be similar to the one being developed, for instance, can benefit a project's planning and budget.

% For instance, a company may be advised by this 

% both of which have the need for such computations, but probably not many resources.

% Describe how the research will benefit other researchers in the field and in related disciplines. What will be done to ensure that they can benefit? 


\section{Background} \label{background}

\par While there has been extensive research in the field of graph processing systems, work on benchmarking and examining performance of these systems has been limited in certain of its aspects. 

\par One important aspect concerns the way graph processing systems' performance is measured. More specifically, although when proposing systems, such as in \cite{pregel,graphxpaper, powergraph}, authors do showcase their implementations' performance, the cases considered mainly focus on multi-node configurations and do not provide a comparison against single-node settings. Additionally, the experiments conducted tend to feature mostly large datasets, which as seen in \cite{survey} is not always the case.

\par Another limiting factor in current research is the fact that although there have been researchers that have benchmarked graph processing systems \cite{benchmarking-vision, how-well, graphalytics, scalability-cost}, in their work they consider and compare the performance of various systems against each other; instead, our focus is to show how executing tasks on single-node settings is at least as efficient as executing them on multi-node settings, as well as in many cases even better.

\par Lastly, regarding the part of our research that may consider also comparing graph database systems' performance, there has been limited research, such as the work done in \cite{traversal-benchmarking}, since the implementations considered in our work, for instance \cite{emptyheaded}, are very recent.

% \par The research community and companies in the industry have been dealing with immense amounts of data. Today's approaches feature the use of distributed big data processing systems, given that other solutions may be extremely slow in terms of the time needed to process the data or carry out the computational tasks. Graph computation problems account for a big partition of the different challenges that are faced. For this reason, several systems have been built that specialise in heavy graph data processing tasks. The systems that will be considered in our research (\hyperref[graph-proc]{Section 3.3}) can be split into two categories; graph data processing systems, and graph database systems. We will now consider both of these types of systems and discuss some of their aspects that are relevant to our research.

% \subsection{Graph Data Processing Systems}

% TODO

% ***
% such as Apache Flink Gelly \cite{flink}, Apache Spark GraphX \cite{spark,graphxpaper}, Apache Giraph \cite{giraph}, GraphLab \cite{graphlab}, PowerGraph \cite{powergraph}, Naiad \cite{naiad}, etc. 


% \subsection{Graph Database Systems}

% While specialised graph data processing systems offer a solution to graph computation tasks, usually individuals have to familiarise themselves with the programming model that the system follows, as well as spend time to learn its programming interface in order to fulfill their tasks. To tackle interaction with low-level primitive systems, other systems were developed, offering a more high-level solution that didn't involve getting this technical in order to complete a graph computational task. 

% \subsection{Related Research}



% \par Modern applications and services handle a vast amount of data

% The rise of interest in graph data has led the 

% Lately there has been a lot of research around graph data processing due to the fact that 

% Demonstrate a knowledge and understanding of past and current work in the subject area, including relevant references like this \cite{template}.

% \section{Programme and Methodology} \label{methodology}
\section{Methodology} \label{methodology}

\subsection{Approach} \label{approach}

\par The core idea is to perform several graph data processing tasks (\hyperref[algos]{Section 3.4}), with the use of different datasets that range from a hundred thousand to forty million vertices (\hyperref[datasets]{Section 3.2}), on selected graph processing systems (\hyperref[graph-proc]{Section 3.3}). Once we execute all these different experiments, we will compare their results against each other.

\subsection{Datasets} \label{datasets}

\par Given that we aim at covering as many cases as possible, in order to provide broad and useful insights through our work, we consider several datasets with different characteristics; that is, different numbers of nodes, edges, edge distributions, etc. The datasets chosen are publicly available and are commonly used within the graph research field. In \hyperref[dataset-table]{Table 1} we present the datasets we intend to use in our experiments and provide their characteristics.

\bigskip

\begin{table}[H]
	\centering
	\begin{tabular}{|l|r|r|l|}
	\hline
	\multicolumn{1}{|c|}{\textbf{Dataset}} & \multicolumn{1}{c|}{\textbf{Nodes}} & \multicolumn{1}{c|}{\textbf{Edges}} & \multicolumn{1}{c|}{\textbf{Description}} \\ \hline
	\hline
	ego-Gplus \cite{snapnets} & 107,614 & 13,673,453 & Social circles from Google+ \\ \hline
	soc-Pokec \cite{snapnets} & 1,632,803 & 30,622,564 & Pokec online social network \\ \hline
	cit-Patents \cite{snapnets} & 3,774,768 & 16,518,948 & Citation network among US Patents \\ \hline
	twitter \cite{twitter} & 41,700,000 & 1,470,000,000 & Twitter follower network \\ \hline
	\end{tabular}
	\caption{Datasets considered along with their characteristics}
	\label{dataset-table}
\end{table}

\par Additionally, if needed, we might consider generating datasets with different distributions for their edges, for example, uniform, skewed, and others.

\subsection{Graph Processing Systems} \label{graph-proc}

\par Initially, we aim at examining the graph processing systems stated below. Note that the systems are arranged in order of our interest, while those marked with an asterisk (*) are not our main focus and may be omitted in case we are limited by time.

\begin{itemize}

	\item Timely Dataflow \cite{naiad, timelydf}
	\item GraphLab \cite{graphlab}
	\item Apache Flink (Gelly) \cite{flink}
	\item Apache Spark (GraphX) \cite{spark,graphxpaper}
	\item *Apache Giraph \cite{giraph}

\end{itemize}

\par The basis for our selection of graph processing systems was their popularity \cite{survey}, as well as their timeliness.

\par After this, as discussed in \hyperref[feasibility]{Section 1.5}, our secondary goal is to also carry out our tasks using graph database systems. In case we have time to consider this category of systems, we hope to test at least one of the following systems. Again, we order the systems based on how interested we are in examining them, their popularity, and how recent they are.

\begin{itemize}

	\item EmptyHeaded \cite{emptyheaded,emptyheadedgit}
	\item Neo4j \cite{neo4j}
	\item *OrientDB \cite{orientdb}
	\item *ArangoDB \cite{arangodb}

\end{itemize}

\subsection{Algorithms} \label{algos}

\par Similarly to the choice of graph processing systems, our choice of algorithms is based on  \cite{survey}. As such, we will consider the following types of computations.

\begin{itemize}
	\item Finding connected components
	\item Neighborhood queries (e.g. finding n-th degree neighbors of a vertex)
	\item Finding short or shortest paths
	\item Subgraph matching (e.g. finding all diamond patterns)
	\item Ranking and centrality scores (e.g. PageRank, Betweeness or Closeness centrality)
	\item Aggregations (e.g. counting the numbers of triangles)
	\item Reachability queries (e.g. checking if vertex $u$ is reachable from vertex $v$)
\end{itemize}

\subsection{Environment Settings}

\par We aim at providing two types of environment settings for our experimentations. Firstly, all experiments will be executed on a single-node setting. Afterwards, depending on the resources, experiments will also be executed in scaled environments, which will consist of multiple nodes. The scaling factor cannot be determined yet since it is dependent on the amount of resources that we will have available to us.

\subsection{Limitations} \label{limitations}

Currently, we are limited by two factors. First, we have not yet gained access to a cluster or distributed environment in order to perform our experiments using the graph data processing systems in their scaled configuration. Then, another factor that limits the scope of our work is time, since the timeline for the completion of our project spans from June to mid August.

% \begin{itemize}
%     \item Detail the methodology to be used in pursuit of the research and justify this choice.
%     \item Describe your contributions and novelty and where you
%     will go beyond the state-of-the-art (new methods, new tools,
%     new data, new insights, new proofs,...)
%     \item Describe the programme of work, indicating the research to be undertaken and the milestones that can be used to measure its progress.
%     \item Where suitable define work packages and define the dependences
%     between these work packages. WPs and their dependences should be
%     shown in the Gantt chart in the research plan.
%     \item Explain how the project will be managed.
%     \item State the limitations of your research.
% \end{itemize}

\section{Evaluation} \label{evaluation}

\par After having completed all steps and tasks mentioned in \hyperref[methodology]{Section 3},we will carry on to analysing and interpreting our results. Our work will mainly focus on measuring the latency (total runtime) of each system to carry out each of the given tasks. Therefore, we define each system's performance as the total number of seconds it took for the system to complete the calculations of a given task.

% \begin{itemize}
%     \item Describe the specific methods of data collection.
%     \item Explain how you intent to analyse and interpret the results.
% \end{itemize}

\section{Expected Outcomes} \label{outcomes}

\par Having completed our work, we aim to show through our results that the achievable performance when executing our tasks with each of the examined graph processing systems on its single-node configuration will be at least as good as the performance on the system's multi-node settings. Such a result will provide a great argument towards avoiding unnecessary resources, or added complexity, when solving graph processing tasks or performing graph data analyses.

% Conclude your research proposal by addressing your predicted outcomes. What are you hoping to prove/disprove? Indicate how you envisage your research will contribute to debates and discussions in your particular subject area:

% \begin{itemize}
%     \item How will your research make an original contribution to knowledge?
%     \item How might it fill gaps in existing work? 
%     \item How might it extend understanding of particular topics?
% \end{itemize}


\section{Milestones, Deliverables and Research Plan} \label{milestones}

\par Finally, we outline our proposed project's preset milestones in \hyperref[table:milestones]{Table 2}, deliverables in \hyperref[table:deliverables]{Table 3}, as well as a timeline of all tasks in \hyperref[fig:gantt]{Figure 1}.


\begin{table}[H]
	\centering
	\begin{center}
			\begin{tabular}{|c|c|p{10cm}|}
			\hline
			\textbf{Milestone} & \textbf{Week} & \textbf{Description} \\
			\hline
			\hline
			$M_1$ & 1 & Completion of background reading \\ \hline
			$M_2$ & 3 & Completion of experiments for each graph processing system in single-node settings \\ \hline
			$M_3$ & 5 & Completion of experiments for each graph processing system in multi-node settings \\ \hline
			$M_4$ & 7 & Completion of experiments with graph database systems \\ \hline
			$M_5$ & 10 & Submission of dissertation \\
			\hline
			\end{tabular} 
	\end{center}
	\caption{Milestones defined in this project.}
	\label{table:milestones}
\end{table}

\begin{table}[H]
	\centering
	\begin{center}
			\begin{tabular}{|c|c|p{11cm}|}
			\hline
			\textbf{Deliverable} & \textbf{Week} & \textbf{Description} \\
			\hline
			\hline
			$D_1$ & 3 & Detailed report of results and statistics of experiments with graph processing systems in single-node configuration \\ \hline
			$D_2$ & 5 & Detailed report of results and statistics of experiments with graph processing systems in multi-node configuration \\ \hline
			$D_3$ & 7 & Detailed report of results and statistics of experiments with graph database systems \\ \hline
			$D_4$ & 10 & Dissertation \\
			\hline
			\end{tabular} 
	\end{center}
	\caption{List of deliverables defined in this project.}
	\label{table:deliverables}
\end{table}


% The project will be completed once the following tasks have been accomplished:
% \begin{itemize}
%     \item Determining all systems that will be examined
%     \item Selecting necessary data sets/algorithms/workloads to perform experiments
%     \item Conducting experiments in a single-core setting, as well as in multi-core settings (eventually scale up)
%     \item Gathering results and comparing performances
    
% \end{itemize}

% Some concerns of the project are the following:
% \begin{itemize}
%     \item Will the experiments produce the desired figures?
%     \item Will there be an obvious difference in performance between the two examined cases?
%     \item Will we have access to more resources (e.g. a cluster)?
% \end{itemize}

\definecolor{barblue}{RGB}{153,204,254}
\definecolor{groupblue}{RGB}{51,102,254}
\definecolor{linkred}{RGB}{165,0,33}
\renewcommand\sfdefault{phv}
\renewcommand\mddefault{mc}
\renewcommand\bfdefault{bc}
\setganttlinklabel{s-s}{START-TO-START}
\setganttlinklabel{f-s}{FINISH-TO-START}
\setganttlinklabel{f-f}{FINISH-TO-FINISH}
% \sffamily

\begin{figure}[H]
\centering
\begin{ganttchart}[
    canvas/.append style={fill=none, draw=black!5, line width=.75pt},
    hgrid style/.style={draw=black!5, line width=.75pt},
    vgrid={*1{draw=black!5, line width=.75pt}},
    % today=7,
    % today rule/.style={
    %   draw=black!64,
    %   dash pattern=on 3.5pt off 4.5pt,
    %   line width=1.5pt
    % },
    % today label font=\small\bfseries,
    title/.style={draw=none, fill=none},
    title label font=\bfseries\footnotesize,
    title label node/.append style={below=7pt},
    include title in canvas=false,
    bar label font=\mdseries\small\color{black!70},
    % bar label node/.append style={left=2cm},
    bar/.append style={draw=none, fill=black!63},
    bar incomplete/.append style={fill=barblue},
    bar progress label font=\mdseries\footnotesize\color{black!70},
    group incomplete/.append style={fill=groupblue},
    group left shift=0,
    group right shift=0,
    group height=.5,
    group peaks tip position=0,
    % group label node/.append style={left=.6cm},
    group progress label font=\bfseries\small,
    link/.style={-latex, line width=1.5pt, linkred},
    link label font=\scriptsize\bfseries,
    link label node/.append style={below left=-2pt and 0pt}
  ]{1}{11}
  \gantttitle[
    title label node/.append style={below left=7pt and 5pt}
  ]{WEEK: \quad}{0}
  \gantttitlelist{1,...,11}{1} \\
	% \ganttgroup{Background Reading}{1}{1} \\
	\ganttbar[
    name=bckg
  ]{Background Reading}{1}{1} \\
	\ganttgroup{Main Experimentation}{2}{5} \\
	\ganttbar[
    name=single
  ]{Single-Node Config.}{2}{3} \\
  \ganttbar[
    name=multi
	]{Multi-Node Config.}{4}{5} \\
	\ganttgroup{Secondary Experimentation}{6}{7} \\
	\ganttbar[
    name=db
	]{Graph Database Systems}{6}{7} \\
	\ganttgroup{Dissertation}{8}{11} \\
	\ganttbar[
    name=results
	]{Evaluation \& Comparison of Results}{8}{8} \\
	\ganttbar[
    name=finish
	]{Documentation and Report}{9}{11}
	\newganttlinktypealias{straight}{f-s}
	\setganttlinklabel{straight}{}
	\ganttlink[link type=straight]{single}{multi}
	\ganttlink[link type=straight]{results}{finish}
  % \ganttbar[
  %   progress=50,
  %   name=WBS1C
  % ]{\textbf{WBS 1.3} Activity C}{4}{10} \\
  % \ganttbar[
  %   progress=0,
  %   name=WBS1D
  % ]{\textbf{WBS 1.4} Activity D}{4}{10} \\[grid]
  % \ganttgroup[progress=0]{WBS 2 Summary Element 2}{4}{10} \\
  % \ganttbar[progress=0]{\textbf{WBS 2.1} Activity E}{4}{5} \\
  % \ganttbar[progress=0]{\textbf{WBS 2.2} Activity F}{6}{8} \\
  % \ganttbar[progress=0]{\textbf{WBS 2.3} Activity G}{9}{10}
  % \ganttlink[link type=s-s]{WBS1A}{WBS1B}
  % \ganttlink[link type=f-s]{WBS1B}{WBS1C}
  % \ganttlink[
  %   link type=f-f,
  %   link label node/.append style=left
  % ]{WBS1C}{WBS1D}
\end{ganttchart}
\caption{Gantt Chart of the activities defined for this project.}
\label{fig:gantt}
\end{figure}

% \definecolor{barblue}{RGB}{153,204,254}
% \definecolor{groupblue}{RGB}{51,102,254}
% \definecolor{linkred}{RGB}{165,0,33}
% \renewcommand\sfdefault{phv}
% \renewcommand\mddefault{mc}
% \renewcommand\bfdefault{bc}
% \setganttlinklabel{s-s}{START-TO-START}
% \setganttlinklabel{f-s}{FINISH-TO-START}
% \setganttlinklabel{f-f}{FINISH-TO-FINISH}
% % \sffamily


% \begin{figure}[H]
% \begin{ganttchart}[
%     y unit title=0.4cm,
%     y unit chart=0.5cm,
%     vgrid,hgrid,
%     x unit=1.55mm,
%     time slot format=isodate,
%     title/.append style={draw=none, fill=barblue},
%     title label font=\sffamily\bfseries\color{white},
%     title label node/.append style={below=-1.6ex},
%     title left shift=.05,
%     title right shift=-.05,
%     title height=1,
%     bar/.append style={draw=none, fill=groupblue},
%     bar height=.6,
%     bar label font=\normalsize\color{black!50},
%     group right shift=0,
%     group top shift=.6,
%     group height=.3,
%     group peaks height=.2,
%     bar incomplete/.append style={fill=green}
%   ]{2018-06-01}{2018-08-16}
%   \gantttitlecalendar{month=name}\\
%   \ganttbar[
%     progress=100,
%     bar progress label font=\small\color{barblue},
%     bar progress label node/.append style={right=4pt},
%     bar label font=\normalsize\color{barblue},
%     name=pp
%   ]{Background Reading}{2018-06-01}{2018-06-14} \\
% \ganttset{progress label text={}, link/.style={black, -to}}
% \ganttgroup{Objective 1}{2018-06-14}{2018-06-30} \\
% \ganttbar[progress=4, name=T1A]{Task A}{2018-06-14}{2018-06-21} \\
% \ganttlinkedbar[progress=0]{Task B}{2018-06-21}{2018-06-30} \\
% \ganttgroup{Objective 2}{2018-07-01}{2018-07-14} \\
% \ganttbar[progress=15, name=T2A]{Task A}{2018-07-01}{2018-07-07} \\
% \ganttlinkedbar[progress=0]{Task B}{2018-07-07}{2018-07-14} \\
% \ganttgroup{Dissertation    }{2018-07-14}{2018-08-16} \\
%   \ganttbar[progress=0]{Task A}{2018-07-14}{2018-08-16}
%   \ganttset{link/.style={green}}
%   \ganttlink[link mid=.4]{pp}{T1A}
%   \ganttlink[link mid=.159]{pp}{T2A}
% \end{ganttchart}
% \caption{Gantt Chart of the activities defined for this project.}
% \label{fig:gantt}
% \end{figure}


\bibliographystyle{unsrt}

{\small
\bibliography{references}
}
\end{document}