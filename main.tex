%
%                       This is a basic LaTeX Template
%                       for the Informatics Research Review

\documentclass[a4paper,11pt]{article}
% Add local fullpage and head macros
\usepackage{head,fullpage}     
% Add graphicx package with pdf flag (must use pdflatex)
\usepackage[pdftex]{graphicx}  
% Better support for URLs
\usepackage{url}
% Date formating
\usepackage{datetime}
% For Gantt chart
\usepackage{pgfgantt}
\usepackage{xcolor}
\usepackage[utf8]{inputenc}
\usepackage{hyperref}
\usepackage{float}

\definecolor{barblue}{RGB}{153,204,254}
\definecolor{groupblue}{RGB}{51,102,254}
\definecolor{linkred}{RGB}{165,0,33}

\newdateformat{monthyeardate}{%
  \monthname[\THEMONTH] \THEYEAR}

\parindent=0pt          %  Switch off indent of paragraphs 
\parskip=5pt            %  Put 5pt between each paragraph  
\Urlmuskip=0mu plus 1mu %  Better line breaks for URLs


%                       This section generates a title page
%                       Edit only the following three lines
%                       providing your exam number, 
%                       the general field of study you are considering
%                       for your review, and name of IRR tutor

\newcommand{\examnumber}{B138641}
\newcommand{\field}{Benchmarking Graph Processing Systems}
\newcommand{\tutor}{Pablo León-Villagrá}
\newcommand{\supervisor}{Dr Milos Nikolic}

\begin{document}
\begin{minipage}[b]{110mm}
        {\Huge\bf School of Informatics
        \vspace*{17mm}}
\end{minipage}
\hfill
\begin{minipage}[t]{40mm}               
        \makebox[40mm]{
        \includegraphics[width=40mm]{crest.png}}
\end{minipage}
\par\noindent
    % Centre Title, and name
\vspace*{2cm}
\begin{center}
        \Large\bf Informatics Project Proposal \\
        \Large\bf \field
\end{center}
\vspace*{1.5cm}
\begin{center}
        \bf \examnumber\\
        \monthyeardate\today
\end{center}
\vspace*{5mm}

%
%                       Insert your abstract HERE
%                       
\begin{abstract}

The aim of this project is to examine state-of-the-art graph processing systems and compare their performance and scalability in both local and parallel settings. More specifically, the project focuses on showing that often times scalability comes at a cost, and in many cases such an approach is not worth the time and effort. In our research we will prove this by conducting several experiments that will demonstrate that a single-core set-up outperforms executions on multi-core environments due to the parallelization overhead that incurs.

\end{abstract}

\vspace*{1cm}

\vspace*{3cm}
Date: \today

\vfill
{\bf Tutor:} \tutor\\
{\bf Supervisor:} \supervisor
\thispagestyle{empty}

\newpage

%                                               Through page and setup 
%                                               fancy headings
\setcounter{page}{1}                            % Set page number to 1
\footruleheight{1pt}
\headruleheight{1pt}
\lfoot{\small School of Informatics}
\lhead{Informatics Research Review}
\rhead{- \thepage}
\cfoot{}
\rfoot{Date: \date{\today}}
%
\tableofcontents % Makes Table of Contents

% \newpage

\section{Introduction} \label{introduction}

% Introduce the topic of research and explain its academic and industrial context.

\par Modern computer applications and services have evolved to an extent that they have become difficult to manage. The amount of data handled in such applications is only growing while the industry and the research community have built new systems and infrastructures to be able to keep up with this growth rate. Additionally, applications and services now feature relational data, which are often expressed as graph structures, fact that makes things even more demanding and complicated.

\medskip

\par To extract useful information and perform data analytics tasks, many different graph data processing systems have been developed. Although these systems have been extensively used in various applications, many people argue that their performance is questionable and that there is a massive overhead that incurs due to the way these systems scale.

\medskip

\par Moreover, there is no clear indication of which system is appropriate for each occasion. For this reason, there have been many researchers focusing on creating benchmarking systems in order to evaluate the performance of such systems and observe their strengths and weaknesses.

\medskip

\par While research on benchmarking graph processing systems has provided information about the performance of several graph processing systems, it tends to be quite specific and concerns certain aspects of these frameworks. This project was motivated by the need for a more thorough investigation that covers a broader range of cases and systems. Our research aims at showing that scalability is not always the best approach by performing various experiments under different conditions, that is, measuring performance using many different algorithms, systems and other parameters. The main goal is to demonstrate how in many cases running such tasks in single node settings is more efficient performance-wise than distributed and multi-node settings. Afterwards, if it is feasible time-wise, a second goal is to indicate how some tasks can be carried out with the use of higher-level systems, such as graph database systems.

\medskip

\par We will now break down the problem and briefly mention some of its aspects along with how this project contributes to the domain of graph processing and why it is important that such an investigation is conducted. In \hyperref[background]{Section 2} we will discuss necessary and relevant knowledge to the problem's domain. Afterwards, we consider specific parts of our research and analyse steps in \hyperref[methodology]{Section 3}. We then focus on the evaluation of our results which will be detailed in \hyperref[evaluation]{Section 4} and highlight what we expect to see in our results in \hyperref[outcomes]{Section 5}. Lastly, we state a detailed report of the project's plan, its objectives and the deliverables in \hyperref[milestones]{Section 6}.

% \begin{itemize}
%     \item Establish the general subject area.
%     \item Describe the broad foundations of your study -- provide adequate background for readers.
%     \item Indicate the general scope of your project.
%     \item Provide an overview of the sections that will appear in your proposal (optional).
%     \item Engage the readers.
% \end{itemize}

\subsection{Problem Statement} \label{problem-statement}

\par While there is research on individual graph processing systems that measures performance under certain circumstances, there hasn't been extensive work that exposes a more objective benchmark for these systems' performance, such as one that features the use of a wide variety of algorithms, multiple set-ups that range from one to many cores, as well as more realistic use cases. Our work focuses on these aspects and will try to provide a more broad benchmark that will be able to better indicate if such systems are worth using and, if so, under which settings.

% \begin{itemize}
%     \item Answer the question:''What is the gap that needs to be filled?"
%     and/or ''What is the problem that needs to be solved?"
%     \item State the problem clearly early in a paragraph.
%     \item Limit the variables you address in stating your problem.
%     \item Consider bordering the problem as a question.
% \end{itemize}

\subsection{Research Hypothesis and Objectives} \label{hypothesis}

\par Our hypothesis is that graph processing systems do not scale as well as they claim to do for common tasks. We were able to identify these common tasks and other useful information form a survey \cite{survey}. Moreover, this survey revealed how the graph research community and the industry perform their analysis tasks, as well as what data they use, what kind of algorithms they run, and so forth; all these were very helpful for formulating our objectives.

\medskip

\par We aim to support our hypothesis by conducting experiments that will demonstrate how the selected graph processing systems perform in single-core settings, as well as when scaled up, and hope to show through our results that for common tasks executions in single-core settings perform better.

\medskip

\par As mentioned in \hyperref[introduction]{Section 1}, a secondary objective of our work will be to provide alternatives to approaches using less primitive systems, such as graph database systems. More specifically, graph processing has become more and more popular and has gained a lot of attention over the past few years. The people that get involved with such tasks are not necessarily researchers of this field or experienced programmers, so the use of these primitive systems can be quite challenging. For this reason, many people use alternatives, instead, such as graph database systems or querying systems to conduct their work. Thus, examining such systems and comparing results against graph processing systems will also provide very useful information and will result to our work covering a broader range of cases, which of course also grows the size of our audience.

% Identify the overall aims of the project and the individual measurable objectives against which you would wish the outcome of the work to be assessed. Clearly spell out any research hypothesis you are following.

% Include a justification (rationale) for the study. Be clear about what your study will not address.

\subsection{Timeliness and Novelty}

\par Advances in technology nowadays have become more sophisticated and involve an extensive amount of data processing, a lot of which is expressed via graph representations. The popularity of this domain is constantly growing, given that more and more researchers and companies analyse graph data and perform graph processing tasks. Consequently, carrying out this research at this point is crucial since it will provide insights and indications as to how analyses and graph processing tasks can be performed, which will greatly impact the way researchers and developers approach their problems.

% Explain why the proposed research is of sufficient timeliness and novelty

\subsection{Significance}

% Having an indication as to what resources an individual may need before performing their graph data analysis tasks or research is of great significance. 


\par The more we know, the better we understand and the better we decide. As such, possessing information related to various aspects of a researcher's or developer's approach for graph data processing tasks is of great significance. The reason for this is that a thorough experimentation that covers many use cases, some of which will probably be similar to the intended task, will help in the process of the decision making with regards to the chosen technologies that will be used, the resources required, and others. This will result to better approaching a problem and providing more timely and efficient solutions.


% The proposal should demonstrate the originality of your intended research. You should therefore explain why your research is important (for example, by explaining how your research builds on and adds to the current state of knowledge in the field or by setting out reasons why it is timely to research your proposed topic) and providing details of any immediate applications, including further research that might be done to build on your findings.

\subsection{Feasibility} \label{feasibility}

\par While our task seems to be quite feasible, we will nonetheless present some factors that may constitute an obstacle to our research or may limit it to an extent in some of its aspects.

\medskip

\par Firstly, an important factor to take into account regards the configurations of the graph processing systems used. In detail, while the single-core approach is fairly easy to implement, the scaled versions require more attention, as they demand more resources. The difficulty we might face concerns our resources, given that as it is, we they are limited to 8 cores. Obtaining access to a cluster environment or some computational nodes will significantly help our research, although, in case such thing cannot be arranged, our work will be limited to single-core experimentation.

\medskip

\par Afterwards, with regards to the scope of our work, we have already briefly mentioned in \hyperref[introduction]{Section 1} and \hyperref[hypothesis]{Section 1.2} that in case we have enough time we will try to provide alternatives to the means of carrying out the graph processing tasks. Therefore, another part of our work that may or may not be feasible to complete is performing the graph processing tasks by using these graph database systems.

% Comment on the feasibility of the research plans given its limited time frame and resources. Outline your plans for a feasibility study before starting e.g.\ major implementation work.

\subsection{Beneficiaries}

\par The outcome of this work will either implicitly, or even explicitly, impact the graph research community, as well as the industry, where developers and analysts use graph processing systems to perform their data analysis tasks. In detail, given that the results will provide insight as to whether a scaled approach that involves distributed graph processing systems is needed, or if a more simple, single-core, implementation is more preferable, researchers and small to medium sized companies can benefit from possessing such information, because it can potentially result to them saving time, effort, and money. 

\medskip

\par This is due to the fact that scaled approaches demand more resources, which often come at a big cost. Additionally, setting up and managing these resources requires a lot of time and effort. Therefore, having a priori knowledge about the resources needed for a specific task, or knowing the performance of an algorithm that might be similar to the one being developed, for instance, can benefit a project's planning and budget.

% For instance, a company may be advised by this 

% both of which have the need for such computations, but probably not many resources.

% Describe how the research will benefit other researchers in the field and in related disciplines. What will be done to ensure that they can benefit? 


\section{Background} \label{background}

\par TODO.

% \par Modern applications and services handle a vast amount of data

% The rise of interest in graph data has led the 

% Lately there has been a lot of research around graph data processing due to the fact that 

% Demonstrate a knowledge and understanding of past and current work in the subject area, including relevant references like this \cite{template}.

% \section{Programme and Methodology} \label{methodology}
\section{Methodology} \label{methodology}

\par We will now present the way by which our research will be conducted; that is, how we plan to approach the problem stated in \hyperref[problem-statement]{Section 1.1}, as well as the steps that will be followed in order to fulfill our project's purpose.

\subsection{Approach} \label{approach}

\par At first, we discuss on a high level how we aim to solve the problem. The core idea is to perform several graph data processing tasks, with a number of different datasets, using several processing systems. Given that there are numerous different systems, as well as algorithms, the number of cases to examine is extremely large. For this reason, we consider a set of each category, based on the popularity and use of each system, as well as the types of tasks carried out, both of which were retrieved from \cite{survey}.

\medskip

\par We now move on to mentioning specifics with regards to the systems and tasks that will be under consideration in \hyperref[graph-proc]{Section 3.2}  and \hyperref[algos]{Section 3.3}, respectively.


\subsection{Datasets} \label{datasets}

\par Write about datasets used - see EmptyHeaded paper

\medskip

\par Additionally, we might consider generating datasets with different distributions of their edges - uniform, skewed, etc.



\subsection{Graph Processing Systems} \label{graph-proc}

\par Initially, we aim at examining the graph processing systems stated below. Note that they are mentioned in order of interest, that is, ... ***

\begin{itemize}

	\item Timely Dataflow *** add citation
	\item GraphLab *** add citation
	\item Apache Flink (Gelly) \cite{flink}
	\item Apache Spark \cite{spark}
	\item Apache Giraph \cite{giraph}

\end{itemize}

\par The basis for our selection of graph processing systems is their popularity \cite{survey} and ***.

\medskip

\par Afterwards, a secondary goal is to also carry out our tasks using graph database systems. In case we have time to consider this category of systems, we hope to test at least one of the following systems. Again, we order the systems based on how interested we are in testing them ***.

\begin{itemize}

	\item EmptyHeaded \cite{emptyheaded} \cite{emptyheadedgit}
	\item Neo4j \cite{neo4j}
	\item OrientDB \cite{orientdb} (maybe)
	\item ArangoDB \cite{arangodb} (maybe)

\end{itemize}

\subsection{Algorithms} \label{algos}

\par Similarly to the choice of graph processing systems, our choice of algorithms is based on the popularity of the type of task \cite{survey}. As such, we will consider the following types of computations:

*** Add examples for each type of algorithm

\begin{itemize}
	\item Finding connected components
	\item Neighborhood queries
	\item Finding short or shortest paths
	\item Subgraph matching
	\item Ranking and centrality scores
	\item Aggregations
	\item Reachability queries
\end{itemize}

\subsection{Environment Settings}

\par We aim at providing two types of environment settings for our experimentations. Firstly, all experiments will be executed on a single-core setting. Afterwards, depending on the resources, experiments will also be executed in scaled environments, which will consist of multiple nodes. The scaling factor cannot be determined yet since it is dependent on the amount of resources that we will have in our availability.

\subsection{Limitations} \label{limitations}

TODO.

% \begin{itemize}
%     \item Detail the methodology to be used in pursuit of the research and justify this choice.
%     \item Describe your contributions and novelty and where you
%     will go beyond the state-of-the-art (new methods, new tools,
%     new data, new insights, new proofs,...)
%     \item Describe the programme of work, indicating the research to be undertaken and the milestones that can be used to measure its progress.
%     \item Where suitable define work packages and define the dependences
%     between these work packages. WPs and their dependences should be
%     shown in the Gantt chart in the research plan.
%     \item Explain how the project will be managed.
%     \item State the limitations of your research.
% \end{itemize}

\section{Evaluation} \label{evaluation}

\par After having completed all steps and tasks mentioned in \hyperref[methodology]{Section 3},we will carry on to analysing and interpreting our results.

\medskip

\par TODO.

\medskip

\par Question(s): 

\begin{itemize}
	
	\item Provide exact evaluation metrics now?
	\item Should I focus on throughput as well or just \textbf{latency}?	
	
\end{itemize}



% \begin{itemize}
%     \item Describe the specific methods of data collection.
%     \item Explain how you intent to analyse and interpret the results.
% \end{itemize}

\section{Expected Outcomes} \label{outcomes}

\par We will now review what we hope to achieve through our research and discuss how our contribution will impact the graph research community, as well as companies or individual developers that perform graph data processing tasks.

\medskip

\par TODO.

% Conclude your research proposal by addressing your predicted outcomes. What are you hoping to prove/disprove? Indicate how you envisage your research will contribute to debates and discussions in your particular subject area:

% \begin{itemize}
%     \item How will your research make an original contribution to knowledge?
%     \item How might it fill gaps in existing work? 
%     \item How might it extend understanding of particular topics?
% \end{itemize}


\section{Research Plan, Milestones and Deliverables} \label{milestones}

\par TODO.

% The project will be completed once the following tasks have been accomplished:
% \begin{itemize}
%     \item Determining all systems that will be examined
%     \item Selecting necessary data sets/algorithms/workloads to perform experiments
%     \item Conducting experiments in a single-core setting, as well as in multi-core settings (eventually scale up)
%     \item Gathering results and comparing performances
    
% \end{itemize}

% Some concerns of the project are the following:
% \begin{itemize}
%     \item Will the experiments produce the desired figures?
%     \item Will there be an obvious difference in performance between the two examined cases?
%     \item Will we have access to more resources (e.g. a cluster)?
% \end{itemize}

% \begin{figure}[H]
% \begin{ganttchart}[
%     y unit title=0.4cm,
%     y unit chart=0.5cm,
%     vgrid,hgrid,
%     x unit=1.55mm,
%     time slot format=isodate,
%     title/.append style={draw=none, fill=barblue},
%     title label font=\sffamily\bfseries\color{white},
%     title label node/.append style={below=-1.6ex},
%     title left shift=.05,
%     title right shift=-.05,
%     title height=1,
%     bar/.append style={draw=none, fill=groupblue},
%     bar height=.6,
%     bar label font=\normalsize\color{black!50},
%     group right shift=0,
%     group top shift=.6,
%     group height=.3,
%     group peaks height=.2,
%     bar incomplete/.append style={fill=green}
%   ]{2018-06-01}{2018-08-16}
%   \gantttitlecalendar{month=name}\\
%   \ganttbar[
%     progress=100,
%     bar progress label font=\small\color{barblue},
%     bar progress label node/.append style={right=4pt},
%     bar label font=\normalsize\color{barblue},
%     name=pp
%   ]{Background Reading}{2018-06-01}{2018-06-14} \\
% \ganttset{progress label text={}, link/.style={black, -to}}
% \ganttgroup{Objective 1}{2018-06-14}{2018-06-30} \\
% \ganttbar[progress=4, name=T1A]{Task A}{2018-06-14}{2018-06-21} \\
% \ganttlinkedbar[progress=0]{Task B}{2018-06-21}{2018-06-30} \\
% \ganttgroup{Objective 2}{2018-07-01}{2018-07-14} \\
% \ganttbar[progress=15, name=T2A]{Task A}{2018-07-01}{2018-07-07} \\
% \ganttlinkedbar[progress=0]{Task B}{2018-07-07}{2018-07-14} \\
% \ganttgroup{Dissertation    }{2018-07-14}{2018-08-16} \\
%   \ganttbar[progress=0]{Task A}{2018-07-14}{2018-08-16}
%   \ganttset{link/.style={green}}
%   \ganttlink[link mid=.4]{pp}{T1A}
%   \ganttlink[link mid=.159]{pp}{T2A}
% \end{ganttchart}
% \caption{Gantt Chart of the activities defined for this project.}
% \label{fig:gantt}
% \end{figure}

% \begin{table}[H]
%     \begin{center}
%         \begin{tabular}{|c|c|l|}
%         \hline
%         \textbf{Milestone} & \textbf{Week} & \textbf{Description} \\
%         \hline
%         $M_1$ & 2 & Feasibility study completed \\
%         $M_2$ & 5 & First prototype implementation completed \\
%         $M_3$ & 7 & Evaluation completed \\
%         $M_4$ & 10 & Submission of dissertation \\
%         \hline
%         \end{tabular} 
%     \end{center}
%     \caption{Milestones defined in this project.}
%     \label{fig:milestones}
% \end{table}

% \begin{table}[H]
%     \begin{center}
%         \begin{tabular}{|c|c|l|}
%         \hline
%         \textbf{Deliverable} & \textbf{Week} & \textbf{Description} \\
%         \hline
%         $D_1$ & 6 & Software tool for \dots\\
%         $D_2$ & 8 & Evaluation report on \dots\\
%         $D_3$ & 10 & Dissertation \\
%         \hline
%         \end{tabular} 
%     \end{center}
%     \caption{List of deliverables defined in this project.}
%     \label{fig:deliverables}
% \end{table}


%                Now build the reference list
\bibliographystyle{unsrt}   % The reference style
%                This is plain and unsorted, so in the order
%                they appear in the document.

{\small
\bibliography{main}       % bib file(s).
}
\end{document}

